% PLEASE USE THIS FILE AS A TEMPLATE
% Check file iosart2c.tex for more examples
%
% Journal:
%   Journal of Ambient Intelligence and Smart Environments (jaise)
%   Web Intelligence and Agent Systems: An International Journal (wias)
%   Semantic Web: Interoperability, Usability, Applicability (SW)
% IOS Press
% Latex 2e

% options: jaise|wias|sw
% add. options: [seceqn,secfloat,secthm,crcready,onecolumn]

\documentclass{iosart2c}

%\documentclass[sw]{iosart2c}
%\documentclass[wias]{iosart2c}
%\documentclass[jaise]{iosart2c}

\usepackage[T1]{fontenc}
\usepackage{times}%
\usepackage{natbib}% for bibliography sorting/compressing
%\usepackage{endnotes}
%\usepackage{graphics}

% Packages added to the template by the authors:
\usepackage{amsfonts}
%\usepackage{amsmath}
\usepackage{color}
\usepackage{ifpdf}
\usepackage{graphicx}
\ifpdf
  \DeclareGraphicsExtensions{.jpg,.mps,.pdf,.png}
\fi
\usepackage{mathtools}
\usepackage{xspace}
\usepackage{verbatim}

%%%%%%%%%%% Put your definitions here

\newcommand{\amalgame}{Amalgame}
\newcommand{\chip}{CHIP\xspace}
\newcommand{\chr}{Constraint Handling Rules\xspace}
\newcommand{\clisp}{Common Lisp\xspace}
\newcommand{\cliopatria}{Cliopatria\xspace}
\newcommand{\clp}{Constraint Logic Programming\xspace}
\newcommand{\datalog}{Datalog\xspace}
\newcommand{\dbtune}{DBtune\xspace}
\newcommand{\europeana}{Europeana\xspace}
\newcommand{\facet}{/facet\xspace}
\newcommand{\fos}{Free and Open Source\xspace}
\newcommand{\html}{HTML\xspace}
\newcommand{\javascript}{JavaScript\xspace}
\newcommand{\jena}{Jena\xspace}
\newcommand{\js}{JS\xspace}
\newcommand{\json}{JSON\xspace}
\newcommand{\linkeddata}{Linked Data\xspace}
\newcommand{\linkeddataset}{Linked Dataset\xspace}
\newcommand{\linkedopendata}{Linked Open Data\xspace}
\newcommand{\lod}{LOD\xspace}
\newcommand{\lodlaundromat}{LOD Laundromat\xspace}
\newcommand{\multimedian}{MultimediaN\xspace}
\newcommand{\nquads}{N-Quads\xspace}
\newcommand{\ntriples}{N-Triples\xspace}
\newcommand{\pengines}{Pengines\xspace}
\newcommand{\poseidon}{Poseidon\xspace}
\newcommand{\prolog}{Prolog\xspace}
\newcommand{\rdf}{RDF\xspace}
\newcommand{\rdfa}{RDFa\xspace}
\newcommand{\rdflib}{rdflib\xspace}
\newcommand{\rdfs}{rdfs\xspace}
\newcommand{\rdfxml}{RDF/XML\xspace}
\newcommand{\rijksmuseum}{Rijksmuseum\xspace}
\newcommand{\semanticweb}{Semantic Web\xspace}
\newcommand{\sparql}{SPARQL\xspace}
\newcommand{\sw}{SW\xspace}
\newcommand{\swiprolog}{SWI-Prolog\xspace}
\newcommand{\swish}{SWISH\xspace}
\newcommand{\thoughtlab}{Thought Lab\xspace}
\newcommand{\trig}{TriG\xspace}
\newcommand{\turtle}{Turtle\xspace}
\newcommand{\webqr}{WebQR\xspace}
\newcommand{\wilbur}{Wilbur\xspace}
\newcommand{\wilburlong}{Wilbur RDF Toolkit\xspace}
\newcommand{\xml}{XML\xspace}
\newcommand{\xmlrdf}{XMLRDF\xspace}
\newcommand{\xsb}{XSB\xspace}

\newcommand{\absolute}[1]{\lvert#1\rvert}
\newcommand{\assignment}[2]{#1 \leftarrow #2}
\newcommand{\bigsetdef}[2]{\big\{#1\,\,\big\vert\,\,#2\big\}}
\newcommand{\cardinality}[1]{\lvert#1\rvert}
\newcommand{\equivpair}[2]{#1 \approx #2}
\newcommand{\equivset}[1]{[#1]_{\approx}}
\newcommand{\interp}[1]{#1^{\mathcal{I}}}
\newcommand{\intersection}[0]{\cap}
\newcommand{\naturalnumbers}[0]{\mathbb{N}}
\newcommand{\pair}[2]{\langle#1,#2\rangle}
\newcommand{\powerset}[1]{\mathcal{P}(#1)}
\newcommand{\range}[2]{#1,\ldots,#2}
\newcommand{\sequence}[2]{\langle#1,\ldots,#2\rangle}
\newcommand{\set}[1]{\{#1\}}
\newcommand{\setdef}[2]{\{#1\,\vert\,#2\}}
\newcommand{\triple}[3]{\langle#1,#2,#3\rangle}
\newcommand{\tuple}[1]{\langle#1\rangle}
\newcommand{\union}[0]{\cup}

\newtheorem{definition}{Definition}
\newtheorem{example}{Example}

%%%%%%%%%%% End of definitions

\pubyear{0000}
\volume{0}
\firstpage{1}
\lastpage{1}

\begin{document}

\begin{frontmatter}

%\pretitle{}
\title{Escaping Compromise: Causal Theory in Ethical Deliberation}
\runningtitle{}
%\subtitle{}

%\review{}{}{}

\author[A]{\fnms{Wouter} \snm{Beek}}
\author[A]{\fnms{Andrea} \snm{Speijer}}
%\runningauthor{}
\address[A]{VU University Amsterdam}

\begin{abstract}
\end{abstract}

\begin{keyword}
Responsibility, Actual causation, Ethical discussion
\end{keyword}

\end{frontmatter}

%%%%%%%%%%% The article body starts:

\section{Actual causation}

Our interpretation of personal responsibility in terms of causation uses the axiomatization of actual causation as given in \cite{halpern2011}.

Actual causation is introduced in \cite{pearl2000}, providing the following two innovations over earlier work such as \cite{lewis1973}:
\begin{itemize}
\item The use of structural equations as a tool for modeling causality. This is more expressive than the earlier used neuron diagrams.
\item The placement of actual causation within a broader context of causal reasoning and causal inference [?].
\end{itemize}

\subsection{Concepts}

\begin{itemize}
\item \textbf{Conjunctive causation} The value of a variable is determined by taking the minimum of the values of other variables.
\item \textbf{Disjunctive causation} The value of a variable is determined by taking the maximum of the values of other variables.
\item \textbf{Endogenous variable} A variable whose value is ultimately determined by the exogenous variables.
\item \textbf{Exogenous variable} A variable whose value is determined by factors outside the model.
\item \textbf{Intervention} Structural equations express the effects of interventions.
\item \textbf{Justification} Modeling choices made (see Subjectivity) should be justifiable.
\item \textbf{Objectivity} Structural equations describe objective features of the world.
\item \textbf{Subjectivity} The choice of variables and their values is subjective.
\end{itemize}

\subsection{Why do we need actual causation?}

The naive view due to \cite{hume1739} states that causation is based on counterfactual dependence, i.e. definition \ref{def:naive_causation}.
This is sometimes called the \emph{but-for} test.

Naive causation has two problems:
\begin{enumerate}
\item Counterfactual should be considered under contingencies.
      See counterexample \ref{ex:forest_fire}.
\item \textbf{Preemption}: counterfactuals should not be considered under all contingencies.
      See counterexample \ref{ex:suzy_billy}.
\end{enumerate}

\begin{definition}[Naive causation]
\label{def:naive_causation}
$A$ has naively causes $B$ if $A$ counterfactually depends on $B$ have happended and $B$ would not have happened if $A$ would not have happened.
\end{definition}

\begin{example}[Suzy \& Billy]
\label{ex:suzy_billy}
Suzy and Billy both pick up rocks and throw them at a bottle.
Suzy's rock gets there first, shattering the bottle.
Since both throws are perfectly accurate, Billy's would have shattered the bottle had Suzy not thrown.
Thus, according to the naive counterfactual definition, Suzy's throw is not a cause of the bottle shattering.
\end{example}

\subsection{Axiomatization}

According to the theory of actual causation, $A$ causes $B$ if $B$ counterfactually depends on $A$ under some contingency that is at least as normal as the actual world.
In example \ref{ex:suzy_billy} the contingency is that Billy does not throw.

\begin{example}[Late arrival]
\label{ex:late_arrival}
If someone typically leaves work at 5:30 PM and arrives home at 6, but, due to unusually bad traffic, arrives home at 6:10, the bad traffic is typically viewed as the cause of his being late, not the fact that he left at 5:30 (rather than 5:20).
\end{example}

\begin{definition}[Signature]
\label{def:signature}
A signature $\mathcal{S}$ is a tuple $\triple{\mathcal{U}}{\mathcal{V}}{\mathcal{R}}$ with $\mathcal{U}$ the set of endogenous variables, $\mathcal{V}$ the set of exogenous variables, and the function $\mathcal{R} : \mathcal{U} \union \mathcal{V} \rightarrow \powerset{\naturalnumbers}$ that assigns possible values to each variable.
\end{definition}

\begin{definition}[Causal model]
\label{def:causal_model}
A causal model $M$ is a pair $\pair{\mathcal{S}}{\mathcal{F}}$ with signature $\mathcal{S}$ and function $\mathcal{F}$ that maps each $X \in \mathcal{V}$ to the function $F_X : (\times_{U \in \mathcal{U}} \mathcal{R}(U)) \times (\times_{Y \in \mathcal{V} \setminus \set{X}} \mathcal{R}(Y)) \rightarrow \mathcal{R}(X)$ determining the value of $X$.
\end{definition}

Example \ref{ex:forest_fire} has the endogenous random variables $F$, $L$, and $A$, and structural equation \ref{eq:forest_fire}.

\begin{example}[Forest fire]
\label{ex:forest_fire}
A forest fire can either be caused by lightning or an arsonist.
\end{example}

\begin{equation}
\label{eq:forest_fire}
F \leftarrow min(L,A)
\end{equation}

\begin{definition}[Altered assignment]
\label{def:altered_assignment}
\[
  \mathcal{F}^{X \leftarrow x}(Y)
:=
  \left\{
    \begin{array}{ll}
      x              & \text{, if $Y = X$}\\
      \mathcal{F}(Y) & \text{, otherwise}
    \end{array}
  \right.
\]
\[
  M_{X \leftarrow x}
:=
  \pair{\mathcal{S}}{\mathcal{F}^{X \leftarrow x}}
\]
\end{definition}

\begin{definition}[Directed path]
\label{def:directed_path}
A directed path $\sequence{X_1}{X_n}$ is a sequence of endogenous variables $X_i$ such that the value of $X_i$ depends on the values of $\sequence{X_1}{X_{i-1}}$.
\end{definition}

Models are assumed to be acyclic.

\begin{definition}[Context]
\label{def:context}
A context $\vec{u}$ is any member of $\times_{U \in \mathcal{U}} \mathcal{R}(U)$.
\end{definition}

\begin{definition}[Primitive event]
\label{def:primitive_event}
A primitive event is a formula of the form $X = x$ with $X \in \mathcal{V}$ and $x \in \mathcal{R}(X)$.
\end{definition}

\begin{definition}[Causal formula]
\label{def:causal_formula}
A causal formula $[\assignment{\vec{Y}}{\vec{y}}]\phi$ is a formula of the form $[\range{\assignment{Y_1}{y_1}}{\assignment{Y_k}{y_k}}]\phi$ where $\phi$ is a Boolean combination of primitive events, $\range{Y_1}{Y_k}$ are distinct variables in $\mathcal{V}$, and $y_i \in \mathcal{R}{Y_i}$.
\end{definition}

A casual formula should be read as stating that $\phi$ is true if variables $Y_i$ are assigned values $y_i$.

$\pair{M}{\vec{u}} \models \psi$ denotes that a casual formula $\psi$ is true in a causal model $M$ given a context $\vec{u}$.

\begin{definition}[Truth]
\begin{itemize}
\item $\pair{M}{\vec{u}} \models V_i = v_i$, if $F_{V_i}(F_{U_1}(U_1), \ldots, F_{U_m}(U_m), F_{V_1}(V_1), \ldots, F_{V_{i-1}}(V_{i-1}), F_{V_{i+1}}(V_{i+1}), \ldots, F_{V_n}(V_n)) = v_i$, where $\mathcal{U} = \set{\range{U_1}{U_m}}$ and $\mathcal{V} = \set{\range{V_1}{V_n}}$.
\item $\pair{M}{\vec{u}} \models \lnot \phi$, if $\pair{M}{\vec{u}} \not\models \phi$.
\item $\pair{M}{\vec{u}} \models \phi \land \psi$, if $\pair{M}{\vec{u}} \models \phi$ and $\pair{M}{\vec{u}} \models \psi$.
\item $\pair{M}{\vec{u}} \models [\assignment{\vec{Y}}{\vec{y}}]\phi$, if $\pair{M_{\assignment{\vec{Y}}{\vec{y}}}}{\vec{u}} \models \phi$.
\end{itemize}
\end{definition}

Actual causes are conjunctions of primitive events.

\begin{definition}[Actual cause]
\label{def:actual_cause}
$\vec{X} = \vec{x}$ is an actual cause of $\phi$ in $\pair{M}{\vec{u}}$ iff
\begin{itemize}
\item $\pair{M}{\vec{u}} \ models (\vec{X} = \vec{x})$ and $\pair{M}{\vec{u}} \models \phi$.
\item There exists a partition of $\mathcal{V}$ into $\vec{Z}$ and $\vec{W}$ with $\vec{X} \subset \vec{Z}$ and a setting $\vec{x'}$ and $\vec{w}$ of the variables in $\vec{X}$ and $\vec{W}$, such that: if $\pair{M}{\vec{u}} \models Z = z^*$ for all $Z \in Z^*$, then:
  \begin{itemize}
  \item $\pair{M}{\vec{u}} \models [\assignment{\vec{X}}{\vec{x'}}, \assignment{\vec{W}}{\vec{w}}] \lnot \phi$.
  \item $\pair{M}{\vec{u}} \models [\assignment{\vec{X}}{\vec{x}}, \assignment{\vec{W'}}{\vec{w}}, \assignment{\vec{Z'}}{\vec{z^*}}] \phi$ for all subsets $\vec{W'}$ of $\vec{W}$ and all subsets $\vec{Z'}$ of $\vec{Z}$, where $\assignment{\vec{W}}{\vec{w}}$ denotes the assignment where the variables in $\vec{W'}$ get the same values are they would in the assignment $\assignment{\vec{W}}{\vec{w}}$.
  \end{itemize}
\item No subset of $\vec{X}$ satisfies the foregoing.
\end{itemize}
\end{definition}

%%%%%%%%%%% The bibliography starts:
\bibliographystyle{plain}
\bibliography{notebook}

\end{document}
